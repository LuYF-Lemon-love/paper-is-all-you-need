
\section*{Author Contributions}
All authors contribute substantially to the MiniCPM project. Shengding Hu lead and participated in all aspects of the projects. This included the scaling experiments (conducted alongside Yuge Tu), babysitting the training of MiniCPM base models, and contributing to various other parts of the research.  Shengding Hu wrote the paper. Chaoqun He was responsible for evaluating MiniCPM, while Ganqu Cui handled the RLHF training. Xiang Long, Zhi Zheng, Xinrong Zhang and Shengding Hu extended the context window to 128K. The MoE research was conducted by  Yewei Fang and Zhi Zheng. Weilin Zhao and Kaihuo Zhang contributed to the training and inference infrastructure. The open-sourcing of MiniCPM was prepared by Yuxiang Huang and Shengding Hu. Shengding Hu, along with Chenyang Zhao, also provided analysis on the WSD scheduler's training dynamics. Zheng Leng Thai developed the tokenizer. The development of MiniCPM-V was carried out by Chongyi Wang and Yuan Yao.
The training corpus of MiniCPM was prepared by Jie Zhou, Jie Cai, Shengding Hu, Zhi Zheng, and Zhongwu Zhai. The paper was proofread by Xingrong Zhang and Chaoqun He.
Insightful instructions on training MiniCPM were provided by Xu Han, Ning Ding, and Zhiyuan Liu. Finally, Zhiyuan Liu, Maosong Sun, Guoyang Zeng, Chao Jia, and Dahai Li offered essential resources for the training of MiniCPM.

\section*{Limitations}
Although we have proposed a thorough study of the scaling law with SLMs, this paper does not extend to training an LLM to validate the scaling law. The application of WSD LRS on LLMs has not been fully explored to date. However, we remain optimistic about its potential advantages.

\section*{Acknowledgement}
MiniCPM was initially published as a technical blog on February 1st, 2024. Since then, we have received numerous insightful feedback from the community, significantly contributing to the development of this paper. We extend our gratitude to Chunting Zhou and Armen Aghajanyan for their valuable discussions. Special thanks go to Peiqin Sun and Yan Wang for their meticulous feedback on clarifying ambiguities in the blog. Additionally, we appreciate the open-source community's efforts in integrating MiniCPM into inference frameworks like llama.cpp, etc. 